\documentclass{report}

\usepackage[utf8]{inputenc}    
\usepackage[T1]{fontenc}
\usepackage[francais]{babel}     

\title{Compte-rendu du jeu du Pendu}
\author{Quentin \bsc{SAUVAGE}}
\date{23 janvier 2017}

\begin{document}
\maketitle
\chapter{Fonctionnement du programme}
Le programme sélectionne un mot au hasard parmi ceux-disponibles dans le fichier "mots.txt". Il demande ensuite à l'utilisateur de choisir une difficulté afin de définir le nombre de vies qu'il possédera ('f' pour neuf vies, 'm' pour sept vies, 'd' pour cinq vies). À chaque tour, l'utilisateur est informé du nombre de vies qu'il lui reste grâce à un message et un dessin dans la console. Il doit ainsi donner une lettre pour dévoiler petit à petit le mot à deviner. Le jeu s'arrête lorsque le joueur n'a plus de vies ou lorsque le mot a été découvert. Le programme demande alors au joueur s'il souhaite commencer une nouvelle partie.
\chapter{Difficultés rencontrées}
\section{Gestion des flux d'entrées}
J'ai eu des difficultés à gérer les données entrées par l'utilisateur puisque j'essayais d'ouvrir le flux d'entrée standard System.in en l'ayant déjà fermé auparavant.

Exemple : Une méthode comportait les lignes suivantes :\\scanner = new Scanner(System.in); \\...\\scanner.close();\\
Il m'était alors impossible d'ouvrir à nouveau le flux d'entrée standard.

J'ai contourné ce problème en passant un scanner en attributs de ma classe Jeu, et je le ferme juste avant l'arrêt du programme.

\section{Gestion des exceptions}
Trois classes ont été créées afin de pouvoir gérer les données de l'entrée standard : ChoixDifficulteException, ChoixRecommencerException et ChoixLettreException. Elles contrôlent respectivement que la difficulté entrée est soit 'f', soit 'm', soit 'd' ; que la réponse donnée à la question "Recommencer ?" est soit 'o', soit 'n' ; et que la réponse donnée à la phrase "Entrer une lettre : " est bien une lettre minuscule (comprise entre 'a' et 'z' inclus donc). Je n'ai pas géré le cas où l'utilisateur appuie sur la touche entrée sans avoir tapé quelque auparavant, dans ce cas, le déroulement du programme n'est pas affecté et le joueur sera invité à redonner une réponse.

\chapter{Ajouts}
\section{Série de victoires}
Le joueur peut consulter à combien de victoires successives il en est. S'il perd, le compteur retombe à zéro, sinon, il augmente de un.
\section{Stockage des mots dans un fichier texte}
Au lieu de stocker les mots en brut dans un tableau de String, le programme récupère une liste de mots stockée dans le fichier mots.txt, à partir de laquelle elle extrait le mot à deviner.

\section{Affichage du pendu}
Pour savoir où en étaient ses vies, le joueur pouvait initialement consulter la console, qui lui affichait un message du type "Vies : x" (x étant le nombre de vies restantes). Cela a été amélioré afin qu'un dessin soit affiché dans la console, évoluant à chaque fois que le joueur perd une vie et l'informant ainsi sur son état.
\end{document}